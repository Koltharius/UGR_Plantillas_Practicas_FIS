% David Sánchez Jiménez
% davidsanchezjimenez@correo.ugr.es

\documentclass[10pt,a4paper,spanish]{report}

\usepackage[spanish]{babel}
\usepackage[utf8]{inputenc}
\usepackage{amsmath, amsthm}
\usepackage{amsfonts, amssymb, latexsym}
\usepackage{enumerate}
\usepackage[official]{eurosym}
\usepackage{graphicx}
\usepackage[usenames, dvipsnames]{color}
\usepackage{colortbl}
\usepackage{multirow}
\usepackage{fancyhdr}
\usepackage{fancybox}
\usepackage{pseudocode}
\usepackage[all]{xy}
\usepackage{minted}
\usepackage{tikz}
\usepackage{pgfplots}

\pgfplotsset{compat=1.5}

% a4large.sty -- fill an A4 (210mm x 297mm) page
% Note: 1 inch = 25.4 mm = 72.27 pt
%       1 pt = 3.5 mm (approx)

% vertical page layout -- one inch margin top and bottom
\topmargin      0 mm    % top margin less 1 inch
\headheight     0 mm    % height of box containing the head
\headsep       10 mm    % space between the head and the body of the page
\textheight   250 mm
\footskip      14 mm    % distance from bottom of body to bottom of foot

% horizontal page layout -- one inch margin each side
%\oddsidemargin    0   mm    % inner margin less one inch on odd pages
%\evensidemargin   0   mm    % inner margin less one inch on even pages
%\textwidth      159.2 mm    % normal width of text on page

\usepackage[math]{iwona}
\usepackage[T1]{fontenc}
\usepackage{inconsolata}

\usepackage[pdftex, bookmarks=true,
bookmarksnumbered=false, % true means bookmarks in
% left window are numbered
bookmarksopen=false,     % true means only level 1
% are displayed.
colorlinks=true,
linkcolor=webblue]{hyperref}

\definecolor{webgreen}{rgb}{0, 0.5, 0} % less intense green
\definecolor{webblue}{rgb}{0, 0, 0.5}  % less intense blue
\definecolor{webred}{rgb}{0.5, 0, 0}   % less intense red
\definecolor{dblackcolor}{rgb}{0.0,0.0,0.0}
\definecolor{dbluecolor}{rgb}{.01,.02,0.7}
\definecolor{dredcolor}{rgb}{0.8,0,0}
\definecolor{dgraycolor}{rgb}{0.30,0.3,0.30}

\newcommand{\HRule}{\rule{\linewidth}{0.5mm}}

\pagestyle{fancy}

\renewcommand{\chaptermark}[1]{%
\markboth{#1}{}}
\renewcommand{\sectionmark}[1]{%
\markright{\thesection\ #1}}
\fancyhf{}
\fancyhead[LE,RO]{\bfseries\thepage}
\fancyhead[LO]{\bfseries\leftmark}
\renewcommand{\headrulewidth}{0.5pt}
\renewcommand{\footrulewidth}{0pt}
\addtolength{\headheight}{0.5pt}
\fancypagestyle{plain}{
\fancyhead{}
\renewcommand{\headrulewidth}{0pt}
}

\usepackage{sectsty}
\chapterfont{\fontfamily{pag}\selectfont}
\sectionfont{\fontfamily{pag}\selectfont}
\subsectionfont{\fontfamily{pag}\selectfont}
\subsubsectionfont{\fontfamily{pag}\selectfont}

\renewcommand{\labelenumi}{\arabic{enumi}. }
\renewcommand{\labelenumii}{\labelenumi\alph{enumii}) }
\renewcommand{\labelenumiii}{\labelenumii\roman{enumiii}: }

\begin{document}

  %%%%%%%%%%%%%%%%%%%%%%%%%%%%%%  1  %%%%%%%%%%%%%%%%%%%%%%%%%%%%%%%%%%%%
  \begin{tabular}{|>{\raggedright}p{58pt}|>{\raggedright}p{109pt}|>{\raggedright}p{1pt}|>{\raggedright}p{17pt}|>{\raggedright}p{28pt}|>{\raggedright}p{0pt}|>{\raggedright}p{18pt}|>{\raggedright}p{20pt}|}

	\hline
  %Actor e Identificador
	\tab \textbf{Actor}\tab  & \multicolumn{5}{p{155pt}|}{Nombre del Actor}	& \multicolumn{2}{p{39pt}|}{\textbf{Identificador}}\tabularnewline

	\hline
  %Descripcion
	\textbf{Descripción} & \multicolumn{7}{p{194pt}|}{Una breve descripción}\tabularnewline

	\hline
  %Caracteristicas
	\textbf{Características} & \multicolumn{7}{p{194pt}|}{Características que describen al actor}\tabularnewline

	\hline
  %Relaciones
	\textbf{Relaciones} & \multicolumn{7}{p{194pt}|}{Relaciones que posee el actor con otros actores del sistema}\tabularnewline
	\hline
  %Referencias
	\textbf{Referencias} & \multicolumn{7}{p{194pt}|}{Elementos del desarrollo en los que interviene el	Actor (Caso de Uso, Diagrama de secuencia, ...}\tabularnewline
	\hline
  %Autor, Fecha y Versión
	\textbf{Autor} & David Sánchez Jiménez \tab  & \multicolumn{2}{p{30pt}|}{
	\textbf{Fecha}} & Fecha & \multicolumn{2}{p{30pt}|}{
	\textbf{Versión}} & 1.0 \tabularnewline
	\hline
	\end{tabular}


	\vspace{0.5cm}	\begin{tabular}{|>{\raggedright}p{61pt}|>{\raggedright}p{190pt}|>{\raggedright}p{61pt}|}
	\hline
  %Atributos
	\tab \multicolumn{3}{|p{313pt}|}{
	\textbf{Atributos}}\tabularnewline
	\hline
  %Nombre, Descripción y Tipo
	\textbf{Nombre}\tab  & \textbf{Descripción}\tab  & \textbf{Tipo}\tabularnewline
	\hline
	 &  & \tabularnewline
	\hline
	 &  & \tabularnewline
	\hline
	 &  & \tabularnewline
	\hline

	\multicolumn{3}{|p{313pt}|}{Listado de los atributos principales del actor, incluyendo
	su nombre, una pequeña descripción del atributo y su tipo}\tabularnewline
	\hline
	\end{tabular}

	\vspace{0.5cm}
	\begin{tabular}{|>{\raggedright}p{337pt}|}
	\hline
  %Comentarios
	\textbf{Comentarios}\tabularnewline
	\hline
	Comentarios adicionales sobre el actor \tabularnewline
	\hline
	\end{tabular}
\end{document}
