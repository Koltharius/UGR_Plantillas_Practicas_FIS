% David Sánchez Jiménez
% davidsanchezjimenez@correo.ugr.es

\documentclass[10pt,a4paper,spanish]{report}

\usepackage[spanish]{babel}
\usepackage[utf8]{inputenc}
\usepackage{amsmath, amsthm}
\usepackage{amsfonts, amssymb, latexsym}
\usepackage{enumerate}
\usepackage[official]{eurosym}
\usepackage{graphicx}
\usepackage[usenames, dvipsnames]{color}
\usepackage{colortbl}
\usepackage{multirow}
\usepackage{fancyhdr}
\usepackage{fancybox}
\usepackage{pseudocode}
\usepackage[all]{xy}
\usepackage{minted}
\usepackage{tikz}
\usepackage{pgfplots}

\pgfplotsset{compat=1.5}

% a4large.sty -- fill an A4 (210mm x 297mm) page
% Note: 1 inch = 25.4 mm = 72.27 pt
%       1 pt = 3.5 mm (approx)

% vertical page layout -- one inch margin top and bottom
\topmargin      0 mm    % top margin less 1 inch
\headheight     0 mm    % height of box containing the head
\headsep       10 mm    % space between the head and the body of the page
\textheight   250 mm
\footskip      14 mm    % distance from bottom of body to bottom of foot

% horizontal page layout -- one inch margin each side
%\oddsidemargin    0   mm    % inner margin less one inch on odd pages
%\evensidemargin   0   mm    % inner margin less one inch on even pages
%\textwidth      159.2 mm    % normal width of text on page

\usepackage[math]{iwona}
\usepackage[T1]{fontenc}
\usepackage{inconsolata}

\usepackage[pdftex, bookmarks=true,
bookmarksnumbered=false, % true means bookmarks in
% left window are numbered
bookmarksopen=false,     % true means only level 1
% are displayed.
colorlinks=true,
linkcolor=webblue]{hyperref}

\definecolor{webgreen}{rgb}{0, 0.5, 0} % less intense green
\definecolor{webblue}{rgb}{0, 0, 0.5}  % less intense blue
\definecolor{webred}{rgb}{0.5, 0, 0}   % less intense red
\definecolor{dblackcolor}{rgb}{0.0,0.0,0.0}
\definecolor{dbluecolor}{rgb}{.01,.02,0.7}
\definecolor{dredcolor}{rgb}{0.8,0,0}
\definecolor{dgraycolor}{rgb}{0.30,0.3,0.30}

\newcommand{\HRule}{\rule{\linewidth}{0.5mm}}

\pagestyle{fancy}

\renewcommand{\chaptermark}[1]{
\markboth{#1}{}}
\renewcommand{\sectionmark}[1]{
\markright{\thesection\ #1}}
\fancyhf{}
\fancyhead[LE,RO]{\bfseries\thepage}
\fancyhead[LO]{\bfseries\leftmark}
\renewcommand{\headrulewidth}{0.5pt}
\renewcommand{\footrulewidth}{0pt}
\addtolength{\headheight}{0.5pt}
\fancypagestyle{plain}{
\fancyhead{}
\renewcommand{\headrulewidth}{0pt}
}

\usepackage{sectsty}
\chapterfont{\fontfamily{pag}\selectfont}
\sectionfont{\fontfamily{pag}\selectfont}
\subsectionfont{\fontfamily{pag}\selectfont}
\subsubsectionfont{\fontfamily{pag}\selectfont}

\renewcommand{\labelenumi}{\arabic{enumi}. }
\renewcommand{\labelenumii}{\labelenumi\alph{enumii}) }
\renewcommand{\labelenumiii}{\labelenumii\roman{enumiii}: }

\begin{document}

    \begin{tabular}{|>{\raggedright}p{70pt}|>{\raggedright}p{254.5pt}|}
      \hline
      %Nombre
      \tab \textbf{Nombre} & Nombre de la operacion y sus parámetros \tabularnewline
      \hline
      %Responsabilidad
      \tab \textbf{Responsabilidad} & Descripción informal de las responsabilidades que debe cumplir la operación \tabularnewline
      \hline
      %Tipo
      \tab \textbf{Tipo} & Concepto, clase o interfaz responsable de la operación \tabularnewline
      \hline
      %Notas
      \tab \textbf{Notas} & Notas de diseño, algoritmo \tabularnewline
      \hline
      %Excepciones
      \tab \textbf{Excepciones} & Casos excepcionales \tabularnewline
      \hline
      %Salida
      \tab \textbf{Salida} & Mensajes o datos que proporciona \tabularnewline
      \hline
      %Precondiciones
      \tab \textbf{Precondiciones} & Suposición acerca del estado del sistema o de los objetos del modelo conceptual antes de ejecutar la operación \tabularnewline
      \hline
      %Postcondiciones
      \tab \textbf{Postcondiciones} & Estado del sistema o de los objetos del modelo conceptual después de la ejecución de la operación \tabularnewline
      \hline
    \end{tabular}
\end{document}
